\documentclass{beamer}

%\mode<presentation>

\usetheme{Dresden}
\usecolortheme{beaver}
\setbeamercovered{transparent}

\usepackage[english]{babel}
\usepackage[latin1]{inputenc}
\usepackage{times}
\usepackage[T1]{fontenc}
\usepackage{mathtools}
% Or whatever. Note that the encoding and the font should match. If T1
% does not look nice, try deleting the line with the fontenc.
\usepackage{amsmath}

\newcommand{\linespace}{\vskip 0.25cm}

%\definecolor{MyForestGreen}{rgb}{0,0.7,0} 
%\newcommand{\tableemph}[1]{{#1}}
%\newcommand{\tablewin}[1]{\tableemph{#1}}
%\newcommand{\tablemid}[1]{\tableemph{#1}}
%\newcommand{\tablelose}[1]{\tableemph{#1}}

%\definecolor{MyLightGray}{rgb}{0.6,0.6,0.6}
%\newcommand{\tabletie}[1]{\color{MyLightGray} {#1}}

% The text in square brackets is the short version of your title and will be used in the
% header/footer depending on your theme.
\title{Monte Carlo Search Tree and Its Applications}

% Sub-titles are optional - uncomment and edit the next line if you want one.
% \subtitle{Why does sub-tree crossover work?} 

% The text in square brackets is the short version of your name(s) and will be used in the
% header/footer depending on your theme.
\author[Magnuson]{Max Magnuson}

% The text in square brackets is the short version of your institution and will be used in the
% header/footer depending on your theme.
\institute[U of Minn, Morris]
{
  Senior Seminar \\ 
  Division of Science and Mathematics \\
  University of Minnesota, Morris \\
  Morris, Minnesota, USA
}

% The text in square brackets is the short version of the date if you need that.
\date{April 25, 2015}

% Delete this, if you do not want the table of contents to pop up at
% the beginning of each subsection:

%{
%  \begin{frame}<beamer>
%    \frametitle{Outline}
%    \tableofcontents
%  \end{frame}
%}

\begin{document}

\begin{frame}
  \titlepage
\end{frame}

% For a 20-25 minute senior seminar talk you probably want something like:
% - Two or three major sections (other than the summary).
% - At *most* three subsections per section.
% - Talk about 30s to 2min per frame. So there should probably be between
%   15 and 30 frames, all told.

\section{Introduction}

%cite image http://www.britannica.com/EBchecked/topic/155485/Deep-Blue
\begin{frame}[fragile]
\frametitle{Kasparov vs Deep Blue}
\begin{figure}
	\includegraphics[height=5.5cm]{Diagrams/KasparovDeepBlue.jpg}
	\centering
\end{figure}
\end{frame}

\begin{frame}
\frametitle{Kasparov vs Deep Blue}
Great display of artifical intelligence (AI) \\
Techniques employed by IBM
\begin{itemize}
	\item Brute force deterministic approach
	\item human knowledge
\end{itemize}
Limitation
\begin{itemize}
	\item scalability into larger search spaces
\end{itemize}
Monte Carlo tree search (MCTS) is an alternative method
\end{frame}

\begin{frame}
  \frametitle{Outline}
  \tableofcontents[] 
\end{frame}

\begin{frame}
\frametitle{Monte Carlo Tree Search (MCTS)}
%cite british Go organisation http://www.britgo.org/learners/chessgo.html
\begin{itemize}
	\item Combines random sampling and game trees
	\item Probabilistic not deterministic
	\item Useful for problems with larger search spaces
\end{itemize}
\end{frame}

\begin{frame}
\frametitle{Applying MCTS to Go}
Go
\begin{itemize}
	\item Board game about positional advantage
	\item Game board for Chess:
	\begin{itemize}
		\item 8x8
	\end{itemize}
	\item Average possible configurations for a game of Chess: 
	\begin{itemize}
		\item 10\textsuperscript{120}
	\end{itemize}
	\item Game board for Go: 
	\begin{itemize}
		\item 19x19
	\end{itemize}
	\item Average possible configurations for a game of Go: 
	\begin{itemize}
		\item 10\textsuperscript{761}
	\end{itemize}
\end{itemize}
\end{frame}


\begin{frame}
\frametitle{Applying MCTS to Narrative Generation}
\begin{itemize}
	\item Useful Applications
	\begin{itemize}
		\item Video game replay value
		\item Educational applications
	\end{itemize}
	\item The search space scales with the number of characters, items, locations, and actions
\end{itemize}
\end{frame}


\section{Naive MCTS Implementation}

\begin{frame}
\frametitle{Outline}
\tableofcontents[currentsection]
\end{frame}

\begin{frame}[fragile]
\frametitle{TicTacToe Diagram}
\begin{figure}[h]
	\includegraphics[width=8.5cm]{Diagrams/TicTacToe/TicTacToeTreeOne.pdf}
	\centering
\end{figure}
\end{frame}

\begin{frame}[fragile]
\frametitle{TicTacToe Diagram}
\begin{figure}[h]
	\includegraphics[width=8.5cm]{Diagrams/TicTacToe/TicTacToeTreeTwo.pdf}
	\centering
\end{figure}
\end{frame}

\begin{frame}[fragile]
\frametitle{TicTacToe Diagram}
\begin{figure}[h]
	\includegraphics[width=8.5cm]{Diagrams/TicTacToe/TicTacToeTreeThree.pdf}
	\centering
\end{figure}
\end{frame}

\begin{frame}[fragile]
\frametitle{TicTacToe Diagram More Levels}
\begin{figure}[h]
	\includegraphics[width=8.5cm]{Diagrams/TicTacToe/TicTacToeTreeMultiLevel.pdf}
	\centering
\end{figure}
\end{frame}

\begin{frame}[fragile]
\frametitle{TicTacToe Diagram}
\begin{figure}[h]
	\includegraphics[width=8.5cm]{Diagrams/TicTacToe/TicTacToeTreeExtended.pdf}
	\centering
\end{figure}
\end{frame}

\begin{frame}[fragile]
\frametitle{TicTacToe Diagram}
\begin{figure}[h]
	\includegraphics[width=10cm]{Diagrams/TicTacToe/TicTacToeTreeExtendedLabeled.pdf}
	\centering
\end{figure}
\end{frame}

\begin{frame}[fragile]
\frametitle{Tree Structure}
\begin{figure}[h]
	\includegraphics[width=10cm]{Diagrams/TicTacToe/MapToTree.pdf}
	\centering
\end{figure}
\end{frame}

\begin{frame}[fragile]
\frametitle{Sampling}
\begin{figure}[h]
	\includegraphics[width=8.5cm]{Diagrams/TicTacToe/TicTacToeTreeSampling.pdf}
	\centering
\end{figure}
\end{frame}

%%%%%%%%%%%%%%%Four Step full diagram begins%%%%%%%%%%%%%%%%%%%%%%%%%%%%

\begin{frame}[fragile]
\frametitle{Four Steps Diagram}
\begin{figure}[h]
	\includegraphics[width=11cm]{Diagrams/FourSteps/MCTSFourStepProcessWhole.pdf}
	\centering
\end{figure}
\end{frame}

\begin{frame}[fragile]
\frametitle{Four Steps Diagram}
\begin{figure}[h]
	\includegraphics[width=11cm]{Diagrams/FourSteps/MCTSFourStepProcessOneOne.pdf}
	\centering
\end{figure}
\end{frame}

\begin{frame}[fragile]
\frametitle{Four Steps Diagram}
\begin{figure}[h]
	\includegraphics[width=11cm]{Diagrams/FourSteps/MCTSFourStepProcessOneTwo.pdf}
	\centering
\end{figure}
\end{frame}

\begin{frame}[fragile]
\frametitle{Four Steps Diagram}
\begin{figure}[h]
	\includegraphics[width=11cm]{Diagrams/FourSteps/MCTSFourStepProcessOneThree.pdf}
	\centering
\end{figure}
\end{frame}

\begin{frame}[fragile]
\frametitle{Four Steps Diagram}
\begin{figure}[h]
	\includegraphics[width=11cm]{Diagrams/FourSteps/MCTSFourStepProcessTwo.pdf}
	\centering
\end{figure}
\end{frame}

\begin{frame}[fragile]
\frametitle{Four Steps Diagram}
\begin{figure}[h]
	\includegraphics[width=11cm]{Diagrams/FourSteps/MCTSFourStepProcessThree.pdf}
	\centering
\end{figure}
\end{frame}

\begin{frame}[fragile]
\frametitle{Four Steps Diagram}
\begin{figure}[h]
	\includegraphics[width=11cm]{Diagrams/FourSteps/MCTSFourStepProcessFourOne.pdf}
	\centering
\end{figure}
\end{frame}

\begin{frame}[fragile]
\frametitle{Four Steps Diagram}
\begin{figure}[h]
	\includegraphics[width=11cm]{Diagrams/FourSteps/MCTSFourStepProcessFourTwo.pdf}
	\centering
\end{figure}
\end{frame}

\begin{frame}[fragile]
\frametitle{Four Steps Diagram}
\begin{figure}[h]
	\includegraphics[width=11cm]{Diagrams/FourSteps/MCTSFourStepProcessFourThree.pdf}
	\centering
\end{figure}
\end{frame}

\begin{frame}[fragile]
\frametitle{Four Steps Diagram}
\begin{figure}[h]
	\includegraphics[width=11cm]{Diagrams/FourSteps/MCTSFourStepProcessFourFour.pdf}
	\centering
\end{figure}
\end{frame}

\begin{frame}[fragile]
\frametitle{Four Steps Diagram}
\begin{figure}[h]
	\includegraphics[width=11cm]{Diagrams/FourSteps/MCTSFourStepProcessWhole.pdf}
	\centering
\end{figure}
\end{frame}

%%%%%%%%%%%%%%%%%%%%%%%%%Four Step diagram ends%%%%%%%%%%%%%%%%%%%%%%%%%%%%%%%%

%%%%%%%%%%%%%%%%%%%%%%%%%First example walkthrough%%%%%%%%%%%%%%%%%%%%%%%%%%%%%%

\begin{frame}[fragile]
\frametitle{Four Steps Diagram}
\begin{figure}[h]
	\includegraphics[width=6.5cm]{Diagrams/MCTSShort/MCTSShortOneOneOne.pdf}
	\centering
\end{figure}
\end{frame}


\begin{frame}[fragile]
\frametitle{Four Steps Diagram}
\begin{figure}[h]
	\includegraphics[width=6.5cm]{Diagrams/MCTSShort/MCTSShortOneOneTwo.pdf}
	\centering
\end{figure}
\end{frame}


\begin{frame}[fragile]
\frametitle{Four Steps Diagram}
\begin{figure}[h]
	\includegraphics[width=6.5cm]{Diagrams/MCTSShort/MCTSShortOneOneThree.pdf}
	\centering
\end{figure}
\end{frame}


\begin{frame}[fragile]
\frametitle{Four Steps Diagram}
\begin{figure}[h]
	\includegraphics[width=6.5cm]{Diagrams/MCTSShort/MCTSShortOneTwo.pdf}
	\centering
\end{figure}
\end{frame}

\begin{frame}[fragile]
\frametitle{Four Steps Diagram}
\begin{figure}[h]
	\includegraphics[width=6.5cm]{Diagrams/MCTSShort/MCTSShortOneThree.pdf}
	\centering
\end{figure}
\end{frame}

\begin{frame}[fragile]
\frametitle{Four Steps Diagram}
\begin{figure}[h]
	\includegraphics[width=6.5cm]{Diagrams/MCTSShort/MCTSShortOneFourOne.pdf}
	\centering
\end{figure}
\end{frame}

\begin{frame}[fragile]
\frametitle{Four Steps Diagram}
\begin{figure}[h]
	\includegraphics[width=6.5cm]{Diagrams/MCTSShort/MCTSShortOneFourTwo.pdf}
	\centering
\end{figure}
\end{frame}

\begin{frame}[fragile]
\frametitle{Four Steps Diagram}
\begin{figure}[h]
	\includegraphics[width=6.5cm]{Diagrams/MCTSShort/MCTSShortOneFourThree.pdf}
	\centering
\end{figure}
\end{frame}

\begin{frame}[fragile]
\frametitle{Four Steps Diagram}
\begin{figure}[h]
	\includegraphics[width=6.5cm]{Diagrams/MCTSShort/MCTSShortOneFourFour.pdf}
	\centering
\end{figure}
\end{frame}

%%%%%%%%%%%%%%%%%%%%%%%%Second example walkthrough%%%%%%%%%%%%%%%%%%%%%%%%%%%%%%%

\begin{frame}[fragile]
\frametitle{Four Steps Diagram}
\begin{figure}[h]
	\includegraphics[width=6.5cm]{Diagrams/MCTSShort/MCTSShortTwoOneOne.pdf}
	\centering
\end{figure}
\end{frame}

\begin{frame}[fragile]
\frametitle{Four Steps Diagram}
\begin{figure}[h]
	\includegraphics[width=6.5cm]{Diagrams/MCTSShort/MCTSShortTwoOneTwo.pdf}
	\centering
\end{figure}
\end{frame}

\begin{frame}[fragile]
\frametitle{Four Steps Diagram}
\begin{figure}[h]
	\includegraphics[width=6.5cm]{Diagrams/MCTSShort/MCTSShortTwoOneThree.pdf}
	\centering
\end{figure}
\end{frame}

\begin{frame}[fragile]
\frametitle{Four Steps Diagram}
\begin{figure}[h]
	\includegraphics[width=6.5cm]{Diagrams/MCTSShort/MCTSShortTwoTwo.pdf}
	\centering
\end{figure}
\end{frame}

\begin{frame}[fragile]
\frametitle{Four Steps Diagram}
\begin{figure}[h]
	\includegraphics[width=6.5cm]{Diagrams/MCTSShort/MCTSShortTwoThree.pdf}
	\centering
\end{figure}
\end{frame}

\begin{frame}[fragile]
\frametitle{Four Steps Diagram}
\begin{figure}[h]
	\includegraphics[width=6.5cm]{Diagrams/MCTSShort/MCTSShortTwoFourOne.pdf}
	\centering
\end{figure}
\end{frame}

\begin{frame}[fragile]
\frametitle{Four Steps Diagram}
\begin{figure}[h]
	\includegraphics[width=6.5cm]{Diagrams/MCTSShort/MCTSShortTwoFourTwo.pdf}
	\centering
\end{figure}
\end{frame}

\begin{frame}[fragile]
\frametitle{Four Steps Diagram}
\begin{figure}[h]
	\includegraphics[width=6.5cm]{Diagrams/MCTSShort/MCTSShortTwoFourThree.pdf}
	\centering
\end{figure}
\end{frame}

\begin{frame}[fragile]
\frametitle{Four Steps Diagram}
\begin{figure}[h]
	\includegraphics[width=6.5cm]{Diagrams/MCTSShort/MCTSShortTwoFourFour.pdf}
	\centering
\end{figure}
\end{frame}

%%%%%%%%%%%%%%%%%%%%%%%%%%Choosing move transition%%%%%%%%%%%%%%%%%%%%%%%%%%%%%

\begin{frame}
\frametitle{What Happens When We Choose a Move?}
Now we have:
\begin{itemize}
	\item{A tree structure}
	\item{A method of generating the tree}
\end{itemize}
What happens when we need to choose a move?
\end{frame}


\begin{frame}[fragile]
\frametitle{Choosing a Move}
\begin{figure}[h]
	\includegraphics[width=6cm]{Diagrams/MakeAMove/MakeAMoveOne.pdf}
	\centering
\end{figure}
\end{frame}

\begin{frame}[fragile]
\frametitle{Choosing a Move}
\begin{figure}[h]
	\includegraphics[width=6cm]{Diagrams/MakeAMove/MakeAMoveSubTrees.pdf}
	\centering
\end{figure}
\end{frame}

\begin{frame}[fragile]
\frametitle{Choosing a Move}
\begin{figure}[h]
	\includegraphics[width=6cm]{Diagrams/MakeAMove/MakeAMoveTwo.pdf}
	\centering
\end{figure}
\end{frame}

\begin{frame}[fragile]
\frametitle{Choosing a Move}
\begin{figure}[h]
	\includegraphics[width=6cm]{Diagrams/MakeAMove/MakeAMoveThree.pdf}
	\centering
\end{figure}
\end{frame}

\begin{frame}
\frametitle{Exploration vs Exploitation}
\begin{itemize}
	\item We might overlook better paths
	\item Exploration vs Exploitation
	\begin{itemize}
		\item Exploration looks at more options
		\item Exploitation focuses on the most promising path
	\end{itemize}
	\item Must find a balance between the two
\end{itemize}
\end{frame}

\begin{frame}[fragile]
\frametitle{Upper Confidence Bound Applied to Trees (UCT)}
\[
	UCT(node)
	{=}
	\alert{\underbrace{\color{black} \frac{W(node)}{N(node)}}_
		{{\text{Value of the Node}}}}
	{+}
	\alert{\underbrace{\color{black} \sqrt[C]{\frac{ln(N(parentNode))}{N(node)}}}_
		{\text{Exploration Bonus}}}
\]
\begin{itemize}
	\item W represents the number of simulated wins
	\item N represents the total number of simulations
	\item C is an experimental constant
	\item Used during tree traversal
	\item Balances exploration vs exploitation
\end{itemize}
\end{frame}

\section{Applying MCTS to Go}

\begin{frame}
\frametitle{Outline}
\tableofcontents[currentsection]
\end{frame}

\begin{frame}
\frametitle{MCTS applied to Go}
What variations can we make specific to Go? \\
In Go each player takes turn placing pieces on a game board
\begin{itemize}
	\item How much does the order of these moves matter?
	\item Can we use this to improve MCTS in the context of Go?
\end{itemize}
\end{frame}

\begin{frame}[fragile]
\frametitle{Tree Redundancy}
\begin{figure}[h]
	\includegraphics[width=11cm]{Diagrams/TicTacToe/MoveOrderNotMattering.pdf}
	\centering
\end{figure}
\end{frame}

\begin{frame}
\frametitle{Rapid Action Value Estimate (RAVE)}
\begin{itemize}
	\item Takes advantage of tree redundancy
	\item Moves have no contextual dependencies
	\item Stores the value of a move with in a subtree at each node
\end{itemize}
\end{frame}

%%%%%%%%%%%%%%%%%%%%%%%%%%%%%%% Start RAVE Diagram %%%%%%%%%%%%%%%%%%%%%%%%%%%%%%%%%%

\begin{frame}[fragile]
\frametitle{RAVE Diagram}
\begin{figure}[h]
	\includegraphics[width=8.5cm]{Diagrams/Rave/RAVEDiagram.pdf}
	\centering
\end{figure}
\end{frame}

\begin{frame}[fragile]
\frametitle{MCTS Values}
\begin{figure}[h]
	\includegraphics[width=8.5cm]{Diagrams/Rave/MCTSValueA.pdf}
	\centering
\end{figure}
\end{frame}

\begin{frame}[fragile]
\frametitle{MCTS Values}
\begin{figure}[h]
	\includegraphics[width=8.5cm]{Diagrams/Rave/MCTSValueB.pdf}
	\centering
\end{figure}
\end{frame}

\begin{frame}[fragile]
\frametitle{RAVE Values}
\begin{figure}[h]
	\includegraphics[width=8.5cm]{Diagrams/Rave/RAVEValueA.pdf}
	\centering
\end{figure}
\end{frame}

\begin{frame}[fragile]
\frametitle{RAVE Values}
\begin{figure}[h]
	\includegraphics[width=8.5cm]{Diagrams/Rave/RAVEValueB.pdf}
	\centering
\end{figure}
\end{frame}

\begin{frame}[fragile]
\frametitle{MCTS RAVE Comparison}
\begin{figure}[h]
	\includegraphics[width=8.5cm]{Diagrams/Rave/RAVEDiagramComparison.pdf}
	\centering
\end{figure}
\end{frame}

%%%%%%%%%%%%%%%%%%%%%%%%%%%%%% End RAVE Diagram %%%%%%%%%%%%%%%%%%%%%%%%%%%%%%%%%%%%%%%%%%%

\begin{frame}
\frametitle{RAVE}
\begin{itemize}
	\item Very powerful approach
	\item Each simulation provides us with more information
	\item Sometimes we do need contextual dependencies
	\begin{itemize}
		\item Example: Close tactical battles
	\end{itemize}
\end{itemize}
\end{frame}

\begin{frame}
\frametitle{MC RAVE}
\begin{itemize}
	\item Combines MCTS values with RAVE values
	\item Uses a weighted average
	\item Favors RAVE values when fewer simulations have been performed
	\begin{itemize}
		\item Contextual dependencies are unknown
	\end{itemize}
	\item Favors MCTS values when more simulations have been performed
	\begin{itemize}
		\item Contextual dependencies are more developed
	\end{itemize}
\end{itemize}
\end{frame}

\begin{frame}
\frametitle{Go Results}
\begin{itemize}
	\item Deterministic approaches could hardly defeat low level amateurs
	\item Computer Go programs use MC RAVE
	\begin{itemize}
		\item MoGo
		\item Crazy Stone
	\end{itemize}
	\item Can compete against top pros in 9x9 Go
	\item Can compete against top pros in handicapped 19x19 Go
\end{itemize}
\end{frame}

\section{Applying MCTS to Narrative Generation}

\begin{frame}
\frametitle{Outline}
\tableofcontents[currentsection]
\end{frame}

\begin{frame}
\frametitle{Narrative Generation}
Kartal et al applied MCTS to Narrative Generation
\begin{itemize}
	\item Crime story
	\item Goals and set up of the story set by the user
	\begin{itemize}
		\item Example Setup: The detective starts in his office
		\item Example Goal: The killer must be arrested
	\end{itemize}
\end{itemize}
Unlike Go and other games
\begin{itemize}
	\item Slightly different tree structure
	\item Evaluation function needed
\end{itemize}
\end{frame}

\begin{frame}
\frametitle{Actions}
\textbf{Move(A, P):} A moves to place P. \\
\textbf{Kill(A, B):} B's health to zero(dead). \\
\textbf{Earthquake(P):} An earthquake strikes at place P. This causes people at P to die (health=0), items to be stuck, and place P to collapse.
\begin{itemize}
	\item Actions drive the story
	\item Different actions take the place of moves as nodes
	\item Set threshold during simulation
\end{itemize}
\end{frame}

\begin{frame}
\frametitle{Evaluation function}
\begin{itemize}
	\item Method of giving nodes value
	\item Ensures stories are interesting
	\item Incorporates believability and goal completion
	\begin{itemize}
		\item Actions are believable based on context
		\begin{itemize}
			\item Example: Inspector searches for clues
			\item Example: Character A kills Character B
		\end{itemize}
		\item Important to complete the goals set by the user
	\end{itemize}
	\item The value is between 0 and 1
	\item Product of every action in a story
\end{itemize}
\end{frame}

\begin{frame}
\frametitle{Narrative Generation Test}
MCTS compared against three deterministic algorithms
\begin{itemize}
	\item Breadth-first search
	\begin{itemize}
		\item Expands tree level by level
	\end{itemize}
	\item Depth-first search
	\begin{itemize}
		\item Expands tree down one path at a time
	\end{itemize}
	\item Best-first search
	\begin{itemize}
		\item Expands tree by choosing the node with the highest estimated value
	\end{itemize}
\end{itemize}
\end{frame}

%%%%%%%%%%%%%%%%%%%%%%%%%%%%%%%%%% Start Breadth-First search %%%%%%%%%%%%%%%%%%%%%%%%%%%%%%%
\begin{frame}[fragile]
\frametitle{Breadth-First Search}
\begin{figure}[h]
	\includegraphics[width=8.5cm]{Diagrams/BreadthFirst/BreadthTreeOne.pdf}
	\centering
\end{figure}
\end{frame}

\begin{frame}[fragile]
\frametitle{Breadth-First Search}
\begin{figure}[h]
	\includegraphics[width=8.5cm]{Diagrams/BreadthFirst/BreadthTreeTwo.pdf}
	\centering
\end{figure}
\end{frame}

\begin{frame}[fragile]
\frametitle{Breadth-First Search}
\begin{figure}[h]
	\includegraphics[width=8.5cm]{Diagrams/BreadthFirst/BreadthTreeThree.pdf}
	\centering
\end{figure}
\end{frame}

\begin{frame}[fragile]
\frametitle{Breadth-First Search}
\begin{figure}[h]
	\includegraphics[width=8.5cm]{Diagrams/BreadthFirst/BreadthTreeFour.pdf}
	\centering
\end{figure}
\end{frame}

\begin{frame}[fragile]
\frametitle{Breadth-First Search}
\begin{figure}[h]
	\includegraphics[width=8.5cm]{Diagrams/BreadthFirst/BreadthTreeFive.pdf}
	\centering
\end{figure}
\end{frame}

\begin{frame}[fragile]
\frametitle{Breadth-First Search}
\begin{figure}[h]
	\includegraphics[width=8.5cm]{Diagrams/BreadthFirst/BreadthTreeSix.pdf}
	\centering
\end{figure}
\end{frame}

%%%%%%%%%%%%%%%%%%%%%%%%%%%%%%%%%% End Breadth-First search %%%%%%%%%%%%%%%%%%%%%%%%%%%%%%%

%%%%%%%%%%%%%%%%%%%%%%%%%%%%%%%%%% Start Depth-First search %%%%%%%%%%%%%%%%%%%%%%%%%%%%%%%
\begin{frame}[fragile]
\frametitle{Depth-First Search}
\begin{figure}[h]
	\includegraphics[width=5cm]{Diagrams/DepthFirst/DepthFirstOne.pdf}
	\centering
\end{figure}
\end{frame}

\begin{frame}[fragile]
\frametitle{Depth-First Search}
\begin{figure}[h]
	\includegraphics[width=4cm]{Diagrams/DepthFirst/DepthFirstTwo.pdf}
	\centering
\end{figure}
\end{frame}

\begin{frame}[fragile]
\frametitle{Depth-First Search}
\begin{figure}[h]
	\includegraphics[width=3cm]{Diagrams/DepthFirst/DepthFirstThree.pdf}
	\centering
\end{figure}
\end{frame}

\begin{frame}[fragile]
\frametitle{Depth-First Search}
\begin{figure}[h]
	\includegraphics[width=2cm]{Diagrams/DepthFirst/DepthFirstFour.pdf}
	\centering
\end{figure}
\end{frame}

\begin{frame}[fragile]
\frametitle{Depth-First Search}
\begin{figure}[h]
	\includegraphics[width=2cm]{Diagrams/DepthFirst/DepthFirstFive.pdf}
	\centering
\end{figure}
\end{frame}

%%%%%%%%%%%%%%%%%%%%%%%%%%%%%%%%%% End Depth-First search %%%%%%%%%%%%%%%%%%%%%%%%%%%%%%%


%%%%%%%%%%%%%%%%%%%%%%%%%%%%%%%%%% Start Best-First search %%%%%%%%%%%%%%%%%%%%%%%%%%%%%%%
\begin{frame}[fragile]
\frametitle{Best-First Search}
\begin{figure}[h]
	\includegraphics[width=8cm]{Diagrams/BestFirst/BestTreeOne.pdf}
	\centering
\end{figure}
\end{frame}

\begin{frame}[fragile]
\frametitle{Best-First Search}
\begin{figure}[h]
	\includegraphics[width=8cm]{Diagrams/BestFirst/BestTreeTwo.pdf}
	\centering
\end{figure}
\end{frame}

\begin{frame}[fragile]
\frametitle{Best-First Search}
\begin{figure}[h]
	\includegraphics[width=8cm]{Diagrams/BestFirst/BestTreeThree.pdf}
	\centering
\end{figure}
\end{frame}

\begin{frame}[fragile]
\frametitle{Best-First Search}
\begin{figure}[h]
	\includegraphics[width=8cm]{Diagrams/BestFirst/BestTreeFour.pdf}
	\centering
\end{figure}
\end{frame}

\begin{frame}[fragile]
\frametitle{Best-First Search}
\begin{figure}[h]
	\includegraphics[width=8cm]{Diagrams/BestFirst/BestTreeFive.pdf}
	\centering
\end{figure}
\end{frame}

%%%%%%%%%%%%%%%%%%%%%%%%%%%%%%%%%% End Best-First search %%%%%%%%%%%%%%%%%%%%%%%%%%%%%%%

\begin{frame}
\frametitle{Test Conditions}
Goals for the narrative:
\begin{itemize}
	\item At least two people are killed
	\item The killer is arrested
\end{itemize}
Each algorithm was given two budgets
\begin{itemize}
	\item 100,000 nodes
	\item 3 million nodes
\end{itemize}
Each algorithm ran three times \\
The score of the narratives were averaged
\end{frame}

\begin{frame}[fragile]
\frametitle{Results}
\begin{table}[h]
\centering
	\begin{tabular}{ c | c | c | c | c |}
	\cline{2-5}	 
	 & MCTS & Breadth-first & Depth-first & Best-first \\ \hline
	\multicolumn{1}{|p{1.5cm}|}{Low Budget} & 0.07 & 0.05 & <0.001 & 0.005 \\ \hline
	\multicolumn{1}{|p{1.5cm}|}{High Budget} & 0.9 & 0.06 & <0.01 & <0.01 \\ \hline
	\end{tabular}
\end{table}
\begin{itemize}
	\item MCTS performed the best in both
	\item Breadth-first came the closest out of the deterministic algorithms
\end{itemize}
\end{frame}

\begin{frame}[fragile]
\frametitle{High Scoring Example Story From MCTS}
\begin{figure}[h]
\begin{tabular}{|p{10cm}|}
\hline
Alice picked up a vase from her house. Bob picked up a rifle from his house. Bob went to Alice's house. While there, greed got the better of him and Bob stole Alice's vase! This made Alice furious. Alice pilfered Bob's vase! This made Bob furious. Bob slayed Alice with a rifle! Bob fled to downtown. Bob executed Inspector Lestrade with a rifle! Charlie took a baseball bat from Bob's house. Sherlock went to Alice's house. Sherlock searched Alice's house and found a clue about the recent crime. Bob fled to Alice's house. Sherlock wrestled the rifle from Bob! This made Bob furious. Sherlock performed a citizen's arrest of Bob with his rifle and took Bob to jail. \\ \hline
\end{tabular}
\centering
\label{fig:GoodStory}
\end{figure}
\end{frame}

\begin{frame}[fragile]
\frametitle{Low Scoring Example from Breadth-First}
\begin{figure}[h]
\begin{tabular}{|p{10cm}|}
\hline
Sherlock moved to Alice's House. An Earthquake occurred at Alice's House! Sherlock and Alice both died due to the earthquake. \\ \hline
\end{tabular}
\centering
\label{fig:BadStory}
\end{figure}
\end{frame}

\section{Conclusion}

\begin{frame}
\frametitle{Outline}
\tableofcontents[currentsection]
\end{frame}

\begin{frame}
\frametitle{Conclusion}
\begin{itemize}
	\item MCTS successful in extending AI capabilities
	\item Tackles problems with larger search spaces
	\item Effective in Go and Narrative Generation
	\item Applicable to other problems
	\begin{itemize}
		\item Can outperform humans in many puzzles
		\item Real time games
		\item Super Mario Brothers
	\end{itemize}
\end{itemize}
\end{frame}

\begin{frame}[fragile]
\frametitle{Any Questions?}
\begin{figure}[h]
	\includegraphics[width=7cm]{Diagrams/NESSuperMarioBros.png}
	\centering
\end{figure}
\end{frame}

\section{Done}
 

\end{document}


