% This is a sample document using the University of Minnesota, Morris, Computer Science
% Senior Seminar modification of the ACM sig-alternate style. Much of this content is taken
% directly from the ACM sample document illustrating the use of the sig-alternate class. Certain
% parts that we never use have been removed to simplify the example, and a few additional
% components have been added.

% See https://github.com/UMM-CSci/Senior_seminar_templates for more info and to make
% suggestions and corrections.

\documentclass{sig-alternate}
\usepackage{color}
%\usepackage[colorinlistoftodos]{todonotes}

%%%%% Uncomment the following line and comment out the previous one
%%%%% to remove all comments
%%%%%\newcommand{\comment}[1]{}
\newcommand{\comment}[1]{{\bf \tt  {#1}}}
%%%%% NOTE: comments still occupy a line even if invisible;
%%%%% Don't write them as a separate paragraph
\newcommand{\todo}[1]{\textcolor{magenta}{\comment{todo: {#1}}}}

\begin{document}

% --- Author Metadata here ---
%%% REMEMBER TO CHANGE THE SEMESTER AND YEAR
\conferenceinfo{UMM CSci Senior Seminar Conference, May 2015}{Morris, MN}

\title{Monte Carlo Tree Search and Its Applications}

\numberofauthors{1}

\author{
% The command \alignauthor (no curly braces needed) should
% precede each author name, affiliation/snail-mail address and
% e-mail address. Additionally, tag each line of
% affiliation/address with \affaddr, and tag the
% e-mail address with \email.
\alignauthor
Max Magnuson\\
	\affaddr{Division of Science and Mathematics}\\
	\affaddr{University of Minnesota, Morris}\\
	\affaddr{Morris, Minnesota, USA 56267}\\
	\email{magnu401@morris.umn.edu}
}

\maketitle
\begin{abstract}

\end{abstract}

\keywords{ACM proceedings, \LaTeX, text tagging}

\section{Introduction} 
\todo{Write stuff in the intro}
\section{Background}
Monte carlo tree search combines the random sampling of traditional monte carlo methods with tree searching. It is important to note that monte carlo tree search is at it's best in games with perfect information. For example, a game like chess has perfect information. All of the information that is needed to play the game is encoded in the game board and pieces, and each player has access to all of that information. In contrast, a game like poker does not have perfect information. Each player does not know the contents of the other player's hands as well as the order of the deck.
\subsection{The Tree Structure}
Monte carlo tree search structures the game state and its potential moves in a tree. Each node in the tree represents the state of the game with the root node representing the current state. Each line represents a legal move that can be made from one game state to another. In other words, it represents the transformation from the parent node to the child node. Any node may have as many children as there are legal moves. For example, at the start of a game of Tic-Tac-Toe the root node may have up to nine children. One for each possible move. Each following child will only be able to have one less child than its parent because the previous moves are no longer available as options. The nodes of the tree also encode for information other than the game state, but we will get into that in the next section.
/todo{I might want to make a graphic to help explain the tree structure with tic-tac-toe}
\subsection{The Four Steps of MCTS}

\subsection{The Upper Confidence Bound(UCT)}

\section{Using MCTS to play Go}

\subsection{Variations in Their MCTS Algorithm}

\subsection{Their Results}

\section{Using MCTS for Narrative Generation}

\subsection{Variations in Their MCTS Algorithm}

\subsection{Their Results}

\section{Using MCTS to play Mario}

\subsection{Variations in Their MCTS Algorithm}

\subsection{Their Results}

\section{Conclusions}

\section{Acknowledgements}

\section{References}

\subsection{Citations}
Citations to articles \cite{Aaronson:2005,Garey:1979,Brun:2008} listed
in the Bibliography section of your
article will occur throughout the text of your article.
You should use BibTeX to automatically produce this bibliography;
you simply need to insert one of several citation commands with
a key of the item cited in the proper location in
the \texttt{.tex} file \cite{OM:2008}.
The key is a short reference you invent to uniquely
identify each work; in this sample document, the key is
the first author's surname and a
word from the title.  This identifying key is included
with each item in the \texttt{.bib} file for your article.

The details of the construction of the \texttt{.bib} file
are beyond the scope of this sample document, but more
information can be found in the \textit{Author's Guide},
and exhaustive details in the \textit{\LaTeX\ User's
Guide}.

This article shows only the plainest form
of the citation command, using \texttt{{\char'134}cite}.
This is what is stipulated in the SIGS style specifications.
No other citation format is endorsed or supported.

% The following two commands are all you need in the
% initial runs of your .tex file to
% produce the bibliography for the citations in your paper.
\bibliographystyle{abbrv}
% sample_paper.bib is the name of the BibTex file containing the
% bibliography entries. Note that you *don't* include the .bib ending here.
\bibliography{sample_paper}  
% You must have a proper ".bib" file
%  and remember to run:
% latex bibtex latex latex
% to resolve all references

\end{document}
