% This is a sample document using the University of Minnesota, Morris, Computer Science
% Senior Seminar modification of the ACM sig-alternate style. Much of this content is taken
% directly from the ACM sample document illustrating the use of the sig-alternate class. Certain
% parts that we never use have been removed to simplify the example, and a few additional
% components have been added.

% See https://github.com/UMM-CSci/Senior_seminar_templates for more info and to make
% suggestions and corrections.

\documentclass{sig-alternate}
\usepackage{color}
%\usepackage[colorinlistoftodos]{todonotes}

%%%%% Uncomment the following line and comment out the previous one
%%%%% to remove all comments
%%%%%\newcommand{\comment}[1]{}
\newcommand{\comment}[1]{{\bf \tt  {#1}}}
%%%%% NOTE: comments still occupy a line even if invisible;
%%%%% Don't write them as a separate paragraph
\newcommand{\todo}[1]{\textcolor{magenta}{\comment{Todo: {#1}}}}

\begin{document}

% --- Author Metadata here ---
%%% REMEMBER TO CHANGE THE SEMESTER AND YEAR
\conferenceinfo{UMM CSci Senior Seminar Conference, May 2015}{Morris, MN}

\title{Monte Carlo Tree Search and Its Applications}

\numberofauthors{1}

\author{
% The command \alignauthor (no curly braces needed) should
% precede each author name, affiliation/snail-mail address and
% e-mail address. Additionally, tag each line of
% affiliation/address with \affaddr, and tag the
% e-mail address with \email.
\alignauthor
Max Magnuson\\
	\affaddr{Division of Science and Mathematics}\\
	\affaddr{University of Minnesota, Morris}\\
	\affaddr{Morris, Minnesota, USA 56267}\\
	\email{magnu401@morris.umn.edu}
}

\maketitle
\begin{abstract}

\end{abstract}

\keywords{ACM proceedings, \LaTeX, text tagging}

\section{Introduction} 

\section{Background}
Monte carlo tree search combines the random sampling of traditional monte carlo methods with tree searching. It is important to note that monte carlo tree search is at it's best in games with perfect information. For example, a game like chess has perfect information. All of the information that is needed to play the game is encoded in the game board and pieces, and each player has access to all of that information. In contrast, a game like poker does not have perfect information. Each player does not know the contents of the other player's hands as well as the order of the deck.
\subsection{The Tree Structure}
Monte carlo tree search structures the game state and its potential moves in a tree. Each node in the tree represents the state of the game with the root node representing the current state. Each line represents a legal move that can be made from one game state to another. In other words, it represents the transformation from the parent node to the child node. Any node may have as many children as there are legal moves. For example, at the start of a game of Tic-Tac-Toe the root node may have up to nine children. One for each possible move. Each following child will only be able to have one less child than its parent since the previous moves are no longer available as options. The nodes of the tree also encode for information other than the game state, but we will get into that in the next section.~\cite{Brand:2014:SEA:2664591.2664612}

\todo{I might want to make a graphic to help explain the tree structure with tic-tac-toe}

\subsection{The Four Steps of MCTS}
The process of monte carlo tree search is split up into four processes: Selection, Expansion, Simulation, and Backpropagation. These four processes are iteratively applied until a decision from the AI must be made.

\subsubsection{Selection}
In the selection process, the MCTS algorithm traverses the current tree looking for the next node to expand. This traversal uses an evaluation function that will prioritize nodes with the highest estimated value. This is known as a tree policy. When a leaf node is reached, it will then transition into expansion.

\subsubsection{Expansion}
In expansion a new node is added to the tree as a child of the leaf reached in the previous step. Then a simulation is played out to determine the value of the newly added node.

\subsubsection{Simulation}
In this step, a simulation is played out according to the simulation policy. The policy randomly selects moves until either an end state or a predefined threshold is reached. Then based on the result of the simulation, the value of the newly added node is established. For example, a simulation of a node for Go would reach the end of a game(the end state), and then determine a value based on whether the player won, drew, or lost.

\subsubsection{Backpropagation}
Once the value of the node is determined, then as the name of this process implies, the rest of the tree can be updated. The algorithm will traverse back to the root node only updating the values of the nodes that it passes through. Only those nodes are effected because each node's respective value is an estimation of values of the nodes after them. For example, a particular node of a tree that encodes for the game Go contains a value that estimates how likely a player is to win or lose given that line of play. In other words, a higher value means that there are more potential games in which that player will win if that move is chosen.

\todo{I should probably introduce the concept of nodes encoding for values earlier in the background when I talk about the tree structure}

\subsection{The Upper Confidence Bound(UCT)}

\section{Using MCTS to play Go}

\subsection{Variations in Their MCTS Algorithm}

\subsection{Their Results}

\section{Using MCTS for Narrative Generation}

\subsection{Variations in Their MCTS Algorithm}

\subsection{Their Results}

\section{Using MCTS to play Mario}

\subsection{Variations in Their MCTS Algorithm}

\subsection{Their Results}

\section{Conclusions}

\section{Acknowledgements}

\section{References}

\subsection{Citations}

% The following two commands are all you need in the
% initial runs of your .tex file to
% produce the bibliography for the citations in your paper.
\bibliographystyle{abbrv}
% sample_paper.bib is the name of the BibTex file containing the
% bibliography entries. Note that you *don't* include the .bib ending here.
\bibliography{MaxMagnusonSeniorSeminar}  
% You must have a proper ".bib" file
%  and remember to run:
% latex bibtex latex latex
% to resolve all references

\end{document}
